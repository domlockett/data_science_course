% Options for packages loaded elsewhere
\PassOptionsToPackage{unicode}{hyperref}
\PassOptionsToPackage{hyphens}{url}
%
\documentclass[
  ignorenonframetext,
]{beamer}
\usepackage{pgfpages}
\setbeamertemplate{caption}[numbered]
\setbeamertemplate{caption label separator}{: }
\setbeamercolor{caption name}{fg=normal text.fg}
\beamertemplatenavigationsymbolsempty
% Prevent slide breaks in the middle of a paragraph
\widowpenalties 1 10000
\raggedbottom
\setbeamertemplate{part page}{
  \centering
  \begin{beamercolorbox}[sep=16pt,center]{part title}
    \usebeamerfont{part title}\insertpart\par
  \end{beamercolorbox}
}
\setbeamertemplate{section page}{
  \centering
  \begin{beamercolorbox}[sep=12pt,center]{part title}
    \usebeamerfont{section title}\insertsection\par
  \end{beamercolorbox}
}
\setbeamertemplate{subsection page}{
  \centering
  \begin{beamercolorbox}[sep=8pt,center]{part title}
    \usebeamerfont{subsection title}\insertsubsection\par
  \end{beamercolorbox}
}
\AtBeginPart{
  \frame{\partpage}
}
\AtBeginSection{
  \ifbibliography
  \else
    \frame{\sectionpage}
  \fi
}
\AtBeginSubsection{
  \frame{\subsectionpage}
}
\usepackage{lmodern}
\usepackage{amssymb,amsmath}
\usepackage{ifxetex,ifluatex}
\ifnum 0\ifxetex 1\fi\ifluatex 1\fi=0 % if pdftex
  \usepackage[T1]{fontenc}
  \usepackage[utf8]{inputenc}
  \usepackage{textcomp} % provide euro and other symbols
\else % if luatex or xetex
  \usepackage{unicode-math}
  \defaultfontfeatures{Scale=MatchLowercase}
  \defaultfontfeatures[\rmfamily]{Ligatures=TeX,Scale=1}
\fi
% Use upquote if available, for straight quotes in verbatim environments
\IfFileExists{upquote.sty}{\usepackage{upquote}}{}
\IfFileExists{microtype.sty}{% use microtype if available
  \usepackage[]{microtype}
  \UseMicrotypeSet[protrusion]{basicmath} % disable protrusion for tt fonts
}{}
\makeatletter
\@ifundefined{KOMAClassName}{% if non-KOMA class
  \IfFileExists{parskip.sty}{%
    \usepackage{parskip}
  }{% else
    \setlength{\parindent}{0pt}
    \setlength{\parskip}{6pt plus 2pt minus 1pt}}
}{% if KOMA class
  \KOMAoptions{parskip=half}}
\makeatother
\usepackage{xcolor}
\IfFileExists{xurl.sty}{\usepackage{xurl}}{} % add URL line breaks if available
\IfFileExists{bookmark.sty}{\usepackage{bookmark}}{\usepackage{hyperref}}
\hypersetup{
  pdftitle={Markdown and Github Pages},
  hidelinks,
  pdfcreator={LaTeX via pandoc}}
\urlstyle{same} % disable monospaced font for URLs
\newif\ifbibliography
\setlength{\emergencystretch}{3em} % prevent overfull lines
\providecommand{\tightlist}{%
  \setlength{\itemsep}{0pt}\setlength{\parskip}{0pt}}
\setcounter{secnumdepth}{-\maxdimen} % remove section numbering

\title{Markdown and Github Pages}
\author{Dominique Lockett\\
\emph{Washington University in St.~Louis}\\
\emph{Department of Politcal Science}}
\date{2020}

\begin{document}
\frame{\titlepage}

\begin{frame}

\end{frame}

\begin{frame}{Orientation for today}
\protect\hypertarget{orientation-for-today}{}

Last time

\begin{enumerate}
\item
  Practiced using API's to access data from online sources
\item
  X
\item
  Y
\end{enumerate}

--

This Time

\begin{enumerate}
\item
  Using Markdown create web content
\item
  Applying your knowledge of Markdown to make yourself a simple website
\end{enumerate}

--

Next time

\begin{enumerate}
\tightlist
\item
  Interactive visualizations

  \begin{itemize}
  \tightlist
  \item
    Use R code to make dynamic web applications that users can interact
    with
  \end{itemize}
\end{enumerate}

\end{frame}

\begin{frame}{Introduction for today}
\protect\hypertarget{introduction-for-today}{}

\begin{enumerate}
\tightlist
\item
  What is Markdown
\end{enumerate}

\begin{itemize}
\item
  Markdown is a powerful yet relatively simple to that allows users to
  create web content
\item
  The language relies on a set of rules which allows you to manage
  headers, font, and lists with ease
\item
  Slack, Reddit and Skype all apply Markdown style rules which make
  formatting messages easy
\end{itemize}

--

\begin{enumerate}
\setcounter{enumi}{1}
\tightlist
\item
  Why should I use it?
\end{enumerate}

\begin{itemize}
\item
  It is an easy-to-read and easy-to-write (and easy-to-learn) plain text
  format
\item
  It is simple to convert to HTML without needing a broad knowledge of
  HTML features which allows you to make a web page without programming
  skills
\item
  Variants such as R Markdown allow you to make complex documents
  (PDF's, word documents, Shiny apps) with ease
\end{itemize}

\end{frame}

\begin{frame}{Syntax}
\protect\hypertarget{syntax}{}

\begin{itemize}
\item
  There are a few features of Markdown which can cover your basic needs
  when presenting text in a variety of formats
\item
  These include:

  \begin{itemize}
  \item
    Text emphasis
  \item
    Headers
  \item
    Lists
  \item
    Paragraphs
  \item
    Hyperlinks
  \item
    Images
  \end{itemize}
\end{itemize}

\end{frame}

\begin{frame}[fragile]{Text Emphasis}
\protect\hypertarget{text-emphasis}{}

\begin{block}{\emph{Italics}}

\begin{itemize}
\tightlist
\item
  \textbf{Code}:
  \texttt{You\ can\ make\ a\ *phrase*\ \_italicized\_\ two\ ways:\ by\ wrapping\ it\ in\ a\ single\ set\ of\ *\ (astericks)\ or\ \_\ (underscores)}
\end{itemize}

--

\begin{itemize}
\tightlist
\item
  \textbf{Output}: You can make a \emph{phrase} \emph{italicized} two
  ways: by wrapping it in a single set of *(astericks) or \_
  (underscores)
\end{itemize}

\end{block}

\begin{block}{\textbf{Bold text}}

\begin{itemize}
\tightlist
\item
  \textbf{Code}:
  \texttt{Similarly\ there\ are\ a\ couple\ of\ ways\ to\ make\ a\ **phrase**\ \_\_bold\_\_:\ by\ using\ two\ sets\ of\ **\ (asterics)\ or\ \_\_\ (underscores)}
\end{itemize}

--

\begin{itemize}
\tightlist
\item
  \textbf{Output}: Similarly there are a couple of ways to make a
  \textbf{phrase} \textbf{bold}: by using two sets of ** (asterics) or
  \_\_ (underscores)
\end{itemize}

\end{block}

\end{frame}

\begin{frame}[fragile]

\begin{columns}[T]
\begin{block}{Headers}

\begin{column}{0.48\textwidth}
\begin{block}{Code}

\begin{itemize}
\item
  \texttt{\#\ First\ level}
\item
  \texttt{\#\#\ Second\ level}
\item
  \texttt{\#\#\#\ Third\ level}
\end{itemize}

--

\end{block}
\end{column}

\begin{column}{0.48\textwidth}
\begin{block}{Output}

\end{block}

\begin{block}{First level}

\begin{block}{Second level}

\begin{block}{Third level}

\end{block}

\end{block}

\end{block}
\end{column}

\end{block}
\end{columns}

\end{frame}

\begin{frame}

\end{frame}

\end{document}
